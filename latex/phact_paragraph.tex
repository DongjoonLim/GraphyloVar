\paragraph{Comparison with PHACT and PHACTboost.}
PHACT~\cite{kim2021phact} and its gradient-boosted extension
PHACTboost~\cite{kim2023phactboost} predict variant pathogenicity using
protein-level features including amino acid conservation, protein structure,
and functional domain annotations. These tools are designed specifically for
missense variants in protein-coding regions and require protein sequence
and structural context as input.

GraphyloVar operates in a fundamentally different scope: it predicts the
functional impact of both coding and non-coding variants using only the
multi-species whole-genome alignment, without any protein-level features.
More than 98\% of the human genome is non-coding, and the majority of
disease-associated variants identified by GWAS fall in non-coding regions
where PHACT and PHACTboost cannot be applied.

The two approaches are therefore complementary rather than competing.
PHACT provides high accuracy for missense variant interpretation by
leveraging protein context, while GraphyloVar provides genome-wide
coverage by learning conservation patterns from the evolutionary tree.
For coding variants, a natural extension would be to combine both
predictions in an ensemble framework, using PHACT for protein-level
features and GraphyloVar for alignment-level phylogenetic features.
We leave this interesting combination for future work.